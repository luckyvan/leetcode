%-*- coding: UTF-8 -*-
% Subsets.tex
% dynamic programming, string
\documentclass[a4paper, titlepage]{article}
\usepackage[tiny]{titlesec} % set section title font
\usepackage{clrscode}
\usepackage{amsmath}
\usepackage{amsthm, enumitem}


%theorem 
\newtheorem{Lemma}{Lemma}

\begin{document}
Given a set of integers, $S$, return all possible subsets. \\

\textbf{Notation and terminology}

We say that $S = \langle x_1, x_2, \dots, x_n \rangle$ is a \emph{set}, in which all $x_i$ are integers and all element placed in non-decending order, that is, $ x_1 \leq x_2 \leq \dots \leq x_n $.


We say that $S'$ is a \emph{subset} of $S$, when $\forall x \in S', x \in S$. In another word,
$S' = \langle x_{i_1}, x_{i_2}, \dots, x_{i_m} \rangle$, in which, $i_1 < i_2 < \dots < i_m \leq n$.

We defines a new set with all possible sets containing only the first \emph{k} elements of $S$.

SUBSETS($S,k$) = \{$\sigma | \sigma = \langle x_{i_1}, x_{i_2}, \dots, x_{i_m} \rangle$, in which $ i_1 < i_2 < \dots < i_m \leq k \leq n$\}

Then our recursive method relies on how to figure out how to calculate the $difference$ of SUBSETS($S, k$), that is,

$\Delta$SUBSETS($S,k$) = SUBSETS($S, k+1$) - SUBSETS($S, k$)

\section{$S$ Without Duplicate Elements}
When $S$ is a set without duplication elements, all elements are in ascending order:
\[
 x_1 < x_2 < \dots < x_n 
\]


\begin{Lemma}\label{lemma: without dup}
$\Delta$ \textup{SUBSETS}($S, k$) = \{$\sigma' | \sigma' = \sigma \cup \{x_{k+1}\}, \sigma \in$ \textup{SUBSETS}($S, k$)\}
\end{Lemma}
\begin{proof}
For $\forall \gamma \in$ SUBSETS($S, k+1$), $\gamma = \langle x_{i_1}, x_{i_2}, \dots, x_{i_m} \rangle$, in which $ i_1 < i_2 < \dots < i_m \leq k + 1 \leq n$.
$\gamma$ can fall into the following two categories.

\begin{enumerate}
\item If $i_m \neq k+1$, then $x_{i_m} \neq x_{k+1}$. Because $x_k \neq x_{k+1}$, $\gamma \in$ SUBSETS($S, k$).

\item If $i_m = k+1$, then $x_{i_m} = x_{k+1}$. Because $x_{i_{m-1}} \neq x_{k+1}$, then $\langle x_{i_1}, x_{i_2}, \dots, x_{i_{m-1}}\rangle \in$ SUBSETS($S, k$). 
\end{enumerate}
\end{proof}

From the Lemma \ref{lemma: without dup}, we now have the following algorithm:

\begin{codebox}
\Procname{$\proc{SUBSETS-WITHOUT-DUPLICATION($S$)}$}
\li $\proc{sort}(S)$
\li $n \gets length(S)$
\li $e \gets empty\ array$ //SUBSETS($S, 0$)
\li $subsets = \{e\}$
\li \For $i \gets 1$ \To $n$
\li 	\Do $subsets \gets subsets + \proc{difference}(subsets, S[i])$
       \End
\li \Return{$subsets$}
\end{codebox}

\begin{codebox}
\Procname{$\proc{DIFFERENCE($subsets, new\_elem$)}$}
\li $n \gets length(subsets)$
\li $subsets' \gets \{\}$
\li \For $i \gets 1$ \To $n$
\li 	\Do $subset' \gets subsets[i] + \{new\_elem\}$
\li 		$subsets' \gets subsets' + \{subset'\}$ 
		\End
\li \Return {$subsets'$}
\end{codebox}
		
\section{$S$ With Duplicate Elements}
When $S$ is a set without duplication elements, all elements are in non-descending order:
\[
 x_1 \leq x_2 \leq \dots \leq x_n 
\]

Unlike Lemma \ref{lemma: without dup}, from current $S$, $(i_m < k \leq n) \not\Rightarrow (x_{i_m} \neq x_k)$. 

\begin{Lemma}\label{lemma: with dup}
With Duplication,
\[
\Delta \textup{SUBSETS(}S, k\textup{)} = \begin{cases}
\{\sigma' | x_k \neq x_{k+1}, \textup{then } \sigma' = \sigma \cup \{x_{k+1}\}, \sigma \in \textup{SUBSETS(}S, k\textup{)}\};  \\
\{\sigma' | x_k = x_{k+1}, \textup{then } \sigma' = \sigma \cup \{x_{k+1}\}, \sigma \in \Delta \textup{SUBSETS(}S, k - 1\textup{)} \}.
\end{cases}
\]
\end{Lemma}
\begin{proof}
For $\forall \gamma \in$ SUBSETS($S, k+1$), $\gamma = \langle x_{i_1}, x_{i_2}, \dots, x_{i_m} \rangle$, in which $ i_1 < i_2 < \dots < i_m \leq k + 1 \leq n$.
When $x_k \neq x_{k+1}$, the discussion of lemma \ref{lemma: without dup} remains valid, so as the 1st part of lemma \ref{lemma: with dup}.

When $x_k = x_{k+1}$, let's say $x_j$ is the last element less than $x_{k+1}$, then SUBSETS($S, j$)  contains all subsets without any $x_{k+1}$. $\Delta\textup{SUBSETS(}S, j\textup{)}$ should be all subsets with exactly 1 $x_{k+1}$. $\Delta\textup{SUBSETS(}S, j+1\textup{)}$ should be all subsets with exactly 2 $x_{k+1}$. 

For $\alpha \in \Delta\textup{SUBSETS(}S, j+1\textup{)}$, let $\alpha'$ be the subset that deduced by getting rid of last element from $\alpha$, then $\alpha' \in \Delta\textup{SUBSETS(}S, j\textup{)}$. Backwards, $\forall \alpha' \in \Delta\textup{SUBSETS(}S, j\textup{)}, \alpha' + \{x_{j+1}\} \in \Delta\textup{SUBSETS(}S, j+1\textup{)}$. From inducation, this applies to the rest element equal to $x_{k+1}$, hence we prove the 2nd part.
\end{proof}

From the Lemma \ref{lemma: with dup}, we now have the following algorithm:

\begin{codebox}
\Procname{$\proc{SUBSETS-WITH-DUPLICATION($S$)}$}
\li $\proc{sort}(S)$
\li $n \gets length(S)$
\li $e \gets empty\ array$ //SUBSETS($S, 0$)
\li $subsets \gets \{e\}$
\li $diffs \gets {}$
\li \For $i \gets 1$ \To $n$
\li 	\Do 
\li 	\If $i = 1$ or $S[i] \neq S[i-1]$
\li 		\Then $diffs[i] \gets \proc{difference}(subsets, S[i])$
\li 		\Else $diffs[i] \gets \proc{difference}(diffs[i-1], S[i])$
			\End
\li 	$subsets \gets subsets + diffs[i]$
       \End
\li \Return{$subsets$}
\end{codebox}
\end{document}