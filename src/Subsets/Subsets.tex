%-*- coding: UTF-8 -*-
% Subsets.tex
% dynamic programming, string
\documentclass[a4paper, titlepage]{article}
\usepackage[tiny]{titlesec} % set section title font
\usepackage{clrscode}
\usepackage{amsmath}
\usepackage{amsthm, enumitem}
\usepackage{fancybox}
\usepackage[all]{xy}

%theorem 
\newtheorem{Lemma}{Lemma}

\begin{document}
Given a set of integers, $S$, return all possible subsets. \\

\textbf{Notation and terminology}

We say that $S = \langle x_1, x_2, \dots, x_n \rangle$ is a \emph{set}, in which all $x_i$ are integers and all element placed in non-decending order, that is, $ x_1 \leq x_2 \leq \dots \leq x_n $.


We say that $S'$ is a \emph{subset} of $S$, when $\forall x \in S', x \in S$. In another word,


$S' = \langle x_{i_1}, x_{i_2}, \dots, x_{i_m} \rangle$, in which, $i_1 < i_2 < \dots < i_m \leq n$.


\section{$S$ Without Duplicate Elements}
When $S$ is a set without duplication elements, all elements are in ascending order:
\[
 x_1 < x_2 < \dots < x_n 
\]

The following defines a new set with all possible sets containing only the first k elements of $S$.

SUBSETS($S,k$) = \{$\sigma | \sigma = \langle x_{i_1}, x_{i_2}, \dots, x_{i_m} \rangle$, in which $ i_1 < i_2 < \dots < i_m \leq k \leq n$\}

\begin{Lemma}
SUBSETS($S, k+1$) - SUBSETS($S, k$) = \{$\sigma' | \sigma' = \sigma \cup \{x_{k+1}\}, \sigma \in$ SUBSETS($S, k$)\}
\end{Lemma}
\begin{proof}
For $\forall \gamma \in$ SUBSETS($S, k+1$), $\gamma = \langle x_{i_1}, x_{i_2}, \dots, x_{i_m} \rangle$, in which $ i_1 < i_2 < \dots < i_m \leq k + 1 \leq n$.

If $i_m \neq k+1$, then $x_{i_m} \neq x_{k+1}$. Because $x_k \neq x_{k+1}$, $\gamma \in$ SUBSETS($S, k$).

If $i_m = k+1$, then $x_{i_m} = x_{k+1}$. Because $x_{i_{m-1}} \neq x_{k+1}$, then $\langle x_{i_1}, x_{i_2}, \dots, x_{i_{m-1}}\rangle \in$ SUBSETS($S, k$). 

\end{proof}
		
\section{$S$ With Duplicate Elements}



\newtheorem*{recursive}{Recursive Formula}
\begin{recursive}
Let us define $c[i,j]$ to be the length of an \emph{longest common subsequence} of $X_i$ and $Y_j$.
If either $i=0$ or $j=0$, one of the sequences has length 0, and so LCS has length 0. Finally, if $X_m$ is a subsequence of $Y_n$, $c[m,n] = m$.
\begin{math}
\\
c[i,j] = 
	\begin{cases}
		0,& \text{if } i = 0 \text{ or } j = 0, \\
		c[i-1, j-1], & \text{if } i,j > 0 \text{ and } x_i = y_j, \\
		\text{max}(c[i, j-1], c[i-1, j]), & \text{if } i,j>{0} \text{ and } x_i \neq y_j, 
	\end{cases}
\end{math}

The following table illustracts the constructed $c[i,j]$ table with $X$ as \emph{``rabbit''} and $Y$ as \emph{``rabbbit''}. The arrows within will be used in the next section of reconstructing the common sequence.

\newcommand{\nar}[1]{\xymatrix@R=3ex@C=3ex{&\\& #1}}
\newcommand{\lular}[1]{\xymatrix@R=3ex@C=3ex{&\\ & #1 \ar[lu]\ar[l]}}
\newcommand{\lar}[1]{\xymatrix@R=3ex@C=3ex{&\\ & #1 \ar[l]}}
\newcommand{\luar}[1]{\xymatrix@R=3ex@C=3ex{&\\ & #1 \ar[lu]}}
\newcommand\MC[1]{\multicolumn{1}{c}{#1}}
\begin{tabular}[b]{rr|*{8}{r|}}
	 	& \MC{$j$}& \MC{0} & \MC{1}	& \MC{2}	& \MC{3}	& \MC{4}	& \MC{5}	& \MC{6}	& \MC{7}	  \\ 	
	$i$	& \MC{} & \MC{$y_j$}& \MC{r}	& \MC{a}	& \MC{b}	& \MC{b}	& \MC{b}	& \MC{i}	& \MC{t}	 \\ 	
     \cline{3-10}
	$0$	& $x_i$	 & \nar{0} & \nar{0}	& \nar{0}	& \nar{0}	& \nar{0}	& \nar{0}	& \nar{0}	& \nar{0}	 \\ 	
     \cline{3-10}
	$1$	& r	& \nar{0}	&\luar{1}& \nar{1}	& \nar{1}	& \nar{1}	& \nar{1}	& \nar{1}	& \nar{1}	 \\ 	
     \cline{3-10}
	$2$	& a	& \nar{0}	& \nar{1}	& \luar{2} &  \lar{2}& \nar{2}		& \nar{2}	& \nar{2}	& \nar{2}	 \\ 	
     \cline{3-10}
	$3$	& b & \nar{0} & \nar{1}	& \nar{2}	&\luar{3}	&\lular{3} & \nar{3}		& \nar{3}	& \nar{3}	 \\ 	
     \cline{3-10}
	$4$	& b	& \nar{0} & \nar{1}	& \nar{2}	& \nar{3}	&\luar{4}	&\lular{4}	&4	& \nar{4}	 \\ 	
     \cline{3-10}
	$5$	& i	& \nar{0} & \nar{1}	& \nar{2}	& \nar{3}	& \nar{4}	& \nar{4}	& \luar{5}	& \nar{5}	 \\ 	
     \cline{3-10}
	$6$	& t & \nar{0} & \nar{1}	& \nar{2}	& \nar{3}	& \nar{4}	& \nar{4}	& \nar{5}	& \luar{6}	 \\ 	
      \cline{3-10}
\end{tabular}
\end{recursive}

\newtheorem*{recon}{Reconstruct Solution}
\begin{recon}
A distinct subsequence is a distince path from $c[m,n]$ to $c[0,0]$. For instance, in our case, 
\begin{enumerate}
	\item $[6,7] \rightarrow [5, 6] \rightarrow [4, 5] \rightarrow [4, 4] \rightarrow [3, 3] \rightarrow [2, 2] \rightarrow [1, 1] \rightarrow [0,0]$ 
	\item $[6,7] \rightarrow [5, 6] \rightarrow [4, 5] \rightarrow [3, 4] \rightarrow [3, 3] \rightarrow [2, 2] \rightarrow [1, 1] \rightarrow [0,0]$
	\item $[6,7] \rightarrow [5, 6] \rightarrow [4, 5] \rightarrow [3, 4] \rightarrow [2, 3] \rightarrow [2, 2] \rightarrow [1, 1] \rightarrow [0,0]$
\end{enumerate}
\end{recon}
\end{document}